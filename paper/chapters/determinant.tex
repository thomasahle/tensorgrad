
\chapter{Determinant and Inverses}
\section{Determinant}

It's convenient to write the determinant in tensor notation as
\[\mathrm{det}(A)=
\frac{1}{n!}\,\mathbin{
\begin{tikzpicture}[baseline=(a0.base), inner sep=1pt]
   \node (a0) {$A$};
   \node[right=.5em of a0] (dots) {$\cdots$};
   \node[right=.5em of dots] (a1) {$A$};
   \draw (a0.north) -- ++(0,.2) coordinate (a0top);
   \draw (a1.north) -- ++(0,.2) coordinate (a1top);
   \draw (a0.south) -- ++(0,-.2) coordinate (a0bot);
   \draw (a1.south) -- ++(0,-.2) coordinate (a1bot);
   \draw[line width=2pt] (a0top -| a0.west) -- (a1top -| a1.east);
   \draw[line width=2pt] (a0bot -| a0.west) -- (a1bot -| a1.east);
\end{tikzpicture}}
\]
where
$\mathbin{
\begin{tikzpicture}[baseline=(a0.south)-3em, inner sep=1pt]
   \node (a0) {\tiny $i_1$};
   \node[right=.5em of a0] (a1) {\tiny $i_2$};
   \node[right=.5em of a1] (dots) {$\dots$};
   \node[right=.5em of dots] (a2) {\tiny $i_n$};
   \draw (a0.south) -- ++(0,-.25) coordinate (a0bot);
   \draw (a1.south) -- ++(0,-.25) coordinate (a1bot);
   \draw (a2.south) -- ++(0,-.25) coordinate (a2bot);
   \draw[line width=2pt] (a0bot -| a0.west) -- (a2bot -| a2.east);
\end{tikzpicture}}
=\varepsilon_{i_1,\dots,i_n}$
is the rank-$n$ Levi-Civita tensor defined by
\[
   \varepsilon_{i_1, \dots, i_n} =
   \begin{cases}
      \mathrm{sign}(\sigma) & \sigma=(i_1, \dots, i_n) \text{ is a permutation} \\
      0 & \text{otherwise.}
   \end{cases}
\]
To see that the definition makes sense, let's first consider
\begin{align*}
   \mathrm{det}(I)=
   \frac{1}{n!}
\mathbin{
   \begin{tikzpicture}[baseline=(dots.base), inner sep=1pt]
      \node (a0) {};
      \node[right=.5em of a0] (a1) {};
      \node[right=.5em of a1] (dots) {$\dots$};
      \node[right=.5em of dots] (a2) {};
      \draw (a0.south) -- ++(0,.3) coordinate (a0top);
      \draw (a1.south) -- ++(0,.3) coordinate (a1top);
      \draw (a2.south) -- ++(0,.3) coordinate (a2top);
      \draw (a0.north) -- ++(0,-.3) coordinate (a0bot);
      \draw (a1.north) -- ++(0,-.3) coordinate (a1bot);
      \draw (a2.north) -- ++(0,-.3) coordinate (a2bot);
      \draw[line width=2pt] (a0top -| a0.west) -- (a2top -| a2.east);
      \draw[line width=2pt] (a0bot -| a0.west) -- (a2bot -| a2.east);
   \end{tikzpicture}
}
%&
=
\frac{1}{n!}
\sum_{i_1, \dots, i_n, j_1, \dots, j_n}
   \varepsilon_{i_1, \dots, i_n}
   \varepsilon_{j_1, \dots, j_n}
   [i=j]
%\\&
=
   \frac{1}{n!}
   \sum_{i_1, \dots, i_n}
   \varepsilon_{i_1, \dots, i_n}^2
%\\&
%=
%   \sum_{i_1, \dots, i_n}
%   [(i_1, \dots, i_n)\text{ is a permutation}]
%\\&
= 1.
\end{align*}
In general we get from the permutation definition of the determinant:
\begin{align*}
   \detstack{A}
   &=
   \sum_{i_1, \dots, i_n, j_1, \dots, j_n}
   \varepsilon_{i_1, \dots, i_n}
   \varepsilon_{j_1, \dots, j_n}
   A_{i_1,j_1} \cdots A_{i_n,j_n}
   \\&=
   \sum_{\sigma, \tau}
   \mathrm{sign}(\sigma)
   \mathrm{sign}(\tau)
   A_{\sigma_1,\tau_1} \cdots A_{\sigma_n,\tau_n}
   \\&=
   \sum_{\sigma}
   \mathrm{sign}(\sigma)
   \sum_\tau
   \mathrm{sign}(\tau)
   A_{\sigma_1,\tau_1} \cdots A_{\sigma_n,\tau_n}
   \\&=
   \sum_{\sigma}
   \mathrm{sign}(\sigma)^2
   \mathrm{det}(A)
   \\&=
   n! \mathrm{det}(A).
\end{align*}
The definition generalizes to Cayley's ``hyper determinants'' by $\dots$.

A curious property is that
\[
   \mathbin{\begin{tikzpicture}[baseline=(a0.base), inner sep=1pt]
      \node (a0) {$A$};
      \node[right=.5em of a0] (dots) {$\cdots$};
      \node[right=.5em of dots] (a1) {$A$};
      \draw (a0.north) -- ++(0,.2) coordinate (a0top);
      \draw (a1.north) -- ++(0,.2) coordinate (a1top);
      \draw (a0.south) -- ++(0,-.2) coordinate (a0bot);
      \draw (a1.south) -- ++(0,-.2) coordinate (a1bot);
      \draw[line width=2pt] (a0bot -| a0.west) -- (a1bot -| a1.east);
   \end{tikzpicture}}
   =
   \mathbin{\begin{tikzpicture}[baseline=(a0.base), inner sep=1pt]
      \node (a0) {$A$};
      \node[right=.5em of a0] (dots) {$\cdots$};
      \node[right=.5em of dots] (a1) {$A$};
      \draw (a0.north) -- ++(0,.2) coordinate (a0top);
      \draw (a1.north) -- ++(0,.2) coordinate (a1top);
      \draw (a0.south) -- ++(0,-.2) coordinate (a0bot);
      \draw (a1.south) -- ++(0,-.2) coordinate (a1bot);
      \draw[line width=2pt] (a0top -| a0.west) -- (a1top -| a1.east);
      \draw[line width=2pt] (a0bot -| a0.west) -- (a1bot -| a1.east);
      \node[above=1.5em of a0] (b0) {$\phantom{A}$};
      \node[above=1.5em of dots] (d1) {$\cdots$};
      \node[above=1.5em of a1] (b1) {$\phantom{A}$};
      \draw (b0.south) -- ++(0,-.2) coordinate (b0bot);
      \draw (b1.south) -- ++(0,-.2) coordinate (b1bot);
      \draw[line width=2pt] (b0bot -| b0.west) -- (b1bot -| b1.east);
   \end{tikzpicture}}
\]

\renewcommand{\arraystretch}{2}
\begin{tabular}[h]{lcc}
   (18) &
   $ \mathrm{det}(A) = \prod_i \lambda_i $
   &
   \dots
   \\
   (19) &
   $ \mathrm{det}(cA) = c^n \mathrm{det}(A) $
   &
   $ \detstack{$cA$} = c^n \detstack{$A$} $
   \\
   (20) &
   $ \mathrm{det}(A) = \mathrm{det}(A^T) $
   &
   \dots
   \\
   (21) &
   $ \mathrm{det}(AB) = \mathrm{det}(A)\mathrm{det}(B) $
   &
   $
   \mathbin{\begin{tikzpicture}[baseline=(a0.base), inner sep=1pt]
      \node (a0) {$B$};
      \node[right=.5em of a0] (dots) {$\cdots$};
      \node[right=.5em of dots] (a1) {$B$};
      \draw (a0.south) -- ++(0,-.2) coordinate (a0bot);
      \draw (a1.south) -- ++(0,-.2) coordinate (a1bot);
      \node[above=.5em of a0] (b0) {$A$};
      \node[above=.5em of dots] (d1) {$\cdots$};
      \node[above=.5em of a1] (b1) {$A$};
      \draw (b0.south) -- (a0.north);
      \draw (b1.south) -- (a1.north);
      \draw (b0.north) -- ++(0,.2) coordinate (b0top);
      \draw (b1.north) -- ++(0,.2) coordinate (b1top);
      \draw[line width=2pt] (b0top -| b0.west) -- (b1top -| b1.east);
      \draw[line width=2pt] (a0bot -| a0.west) -- (a1bot -| a1.east);
   \end{tikzpicture}}
   =
   \mathbin{\begin{tikzpicture}[baseline=(a0.base), inner sep=1pt]
      \node (a0) {$B$};
      \node[right=.5em of a0] (dots) {$\cdots$};
      \node[right=.5em of dots] (a1) {$B$};
      \draw (a0.north) -- ++(0,.2) coordinate (a0top);
      \draw (a1.north) -- ++(0,.2) coordinate (a1top);
      \draw (a0.south) -- ++(0,-.2) coordinate (a0bot);
      \draw (a1.south) -- ++(0,-.2) coordinate (a1bot);
      \draw[line width=2pt] (a0top -| a0.west) -- (a1top -| a1.east);
      \draw[line width=2pt] (a0bot -| a0.west) -- (a1bot -| a1.east);
      \node[above=1.5em of a0] (b0) {$A$};
      \node[above=1.5em of dots] (d1) {$\cdots$};
      \node[above=1.5em of a1] (b1) {$A$};
      \draw (b0.north) -- ++(0,.2) coordinate (b0top);
      \draw (b1.north) -- ++(0,.2) coordinate (b1top);
      \draw (b0.south) -- ++(0,-.2) coordinate (b0bot);
      \draw (b1.south) -- ++(0,-.2) coordinate (b1bot);
      \draw[line width=2pt] (b0top -| b0.west) -- (b1top -| b1.east);
      \draw[line width=2pt] (b0bot -| b0.west) -- (b1bot -| b1.east);
   \end{tikzpicture}}
   $
   \\
   (22) &
   $ \mathrm{det}(A^{-1}) = 1/\mathrm{det}(A) $
   &
   \dots
   \\
   (23) &
   $ \mathrm{det}(A^n) = \mathrm{det}(A)^n $
   &
   \dots
   \\
   (24) &
   $ \mathrm{det}(I+uv^T) = 1 + u^Tv $
   &
   \dots
   \\
\end{tabular}


\section{Inverses}
Might be reduced, unless cofactor matrices have a nice representation?


